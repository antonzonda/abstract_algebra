\documentclass{article}
\usepackage{amsmath}
\usepackage{amsthm}
\usepackage{amsfonts}

\newtheorem{thm}{Theorem}[section]
\newtheorem{lemma}[thm]{Lemma}
\newtheorem{col}[thm]{corollary}

\theoremstyle{definition}
\newtheorem{define}[thm]{Definition}

\theoremstyle{remark}
\newtheorem*{rmk}{Remark}
\newtheorem*{eg}{Example}

\begin{document}
\section{Rings}
\begin{define}[Ring]
  A ring $R$ is a set associated with two operations $+$ and $\cdot$, such that
  it forms a $abelian group$ under $+$. 

  It is $closed$ and $associative$ under $\cdot$, i.e. $\forall a,b \in R$, 
  $a \cdot b \in R$, and $\forall a,b,c \in R, a(bc) = (ab)c$

  Also, it follows $distributive$ laws, that, $\forall a,b,c \in R$, 
  $a\cdot(b+c) = a\cdot b + a\cdot c$
\end{define}
\begin{rmk}
  Some books enforce a multiplicative identity or {\it unit}, denoted as $1$, 
  and refer to rings without unit as `rng'. However, here our ring does not require
  unit.
\end{rmk}
It is easy to come up with various examples of rings, most notably 
$(\mathbb{Z}, +, \cdot), (\mathbb{R}, +, \cdot)$. And we can come up 
with more sophisticated examples like $(C^0[0,1], +, \cdot)$, which is 
the set of all continuous functions defined on $[0,1]$. 

It is clear that the structure of rings are not very restricted. 
For example, commutativity is not required under multiplication.
So we define some `better' rings.
\begin{define}[Integral domain]
  A ring $R$ is called an {\it integral domain}, if for $a, b \in R$ and $ab = 0$, 
  implies $a = 0$ or $b = 0$.
\end{define}

\begin{define}[Division ring]
  A ring $R$ with unit $1$ is a {\it division ring} if $\forall a \neq 0 \in R$, 
  $\exists b$ s.t. $ab = ba = 1$. 
\end{define}

\begin{define}[field]
  A ring $F$ is {\it field} if it is a commutative division ring.
\end{define}
\begin{rmk}
  It is equivalent to say $F$ is a field $\iff$ $F$ is an abelian group under addition
  and $F \setminus \{0\}$ is a group under multiplication.
\end{rmk}

\begin{define}
  An element $a \neq 0$ in a ring $R$ is called {\it zero-divisor} if $\exists b \neq 0$ in 
  $R$ s.t. $ab = 0$.
\end{define}

\begin{define}[subring]
  A $subring$ of a ring $R$ is a subset $S$ s.t. it forms a ring under the $+$ and $\cdot$ 
  of $R$.
\end{define}

\begin{rmk}
  $S$ is a subring if and only if $S$ is closed under addition and multiplication. 
\end{rmk}

Now we have set up the basic definitions, and we shall derive some simple corollaries and lemmas
and show more examples. 

\begin{lemma}
  If $R$ with unit $1$ satisfies all the axioms of rings except $a + b = b + a$, then 
  it is a ring. 
\end{lemma}
\begin{proof}
  Consider $(1 + 1)(a + b) = (a + b) + (a + b)$ and $(1 + 1)(a + b) = (a + a) + (b + b)$ by 
  distributive laws. So $a+b = b+a$.
\end{proof}

\begin{lemma}
  A division ring is an integral domain.
\end{lemma}
\begin{proof}
  Let $R$ be a division ring, take $a,b\in R$ and $ab = 0$. Then, if $a \neq 0$, 
  $a^{-1} (ab) = (a^{-1} a)b = b = 0$. So $a$ or $b$ is $0$.
\end{proof}

\begin{lemma}
  A finite integral domain is a field. 
\end{lemma}
\begin{proof}
  Let $F$ be a finite integral domain, and suppose $|F| = n$. 
  Take non-zero element $a \in F$, then consider $a, a^2, \ldots, a^n$, where 
  we must have $a^i = a^j$ with $1 \leq i < j \leq n$. Then take any $r \in R$, 
  we have $a^i r = a^j r \implies a^i (a^{j-i} r - r) = 0 $, which by the fact that $a^i \neq 0$, 
  implies that $a^{j-i} r = r$. So $a^{j-i} = 1$. 
  
  Therefore, for all non zero $ a \in F$, we have some $k$ s.t. $a^k = 1$, 
  and it implies $a^{k-1} = a^{-1}$ for $k > 1$ (If $k = 1$, then its inverse is itself). 
\end{proof}

\begin{eg}
  We consider $\mathbb{Z}/n\mathbb{Z}$. Note it is a ring, but it is a integral domain 
  if and only if $n$ is prime. Because if $n$ is not prime, say $n = ab$ where $a,b>1$, 
  then $[a][b] = [n] = [0]$ in $\mathbb{Z}/n\mathbb{Z}$. Meanwhile, if $n$ is prime, 
  then for $[a][b] = [0] \iff ab = kn$, which implies $n|a$ or $b|n$ since $n$ prime. 

  Further $\mathbb{Z}/p\mathbb{Z}$ is a field for $p$ prime by the previous lemma.
\end{eg}

We continue to develop some theorems on finite field.
\begin{thm}
  Let $F$ be a finite field, then $|F| = p^n$ for some prime $p$.   
\end{thm}
\begin{proof}
  First we prove there exists a prime $p$ s.t. $\forall a \in F, pa = 0$. 
  Note that there exists such integer like $|F| a = 0$ since it is a group under addition. 
  Let $p$ be the smallest integer s.t. $pa = 0$. Suppose $p$ is not prime, or $p = rs$.
  We have $(rs)a = 0 \implies (rs) (a \cdot 1_R) = 0 \implies (ra) (s\cdot 1_R) = 0$. 
  If $ra = 0$, then $r$ is a smaller integer than $p$, contradict. If $s \cdot 1_R = 0$, then 
  $s = 0$ contradict. So $p$ is prime.

  Next we show $|F| = p^n$ for some $n$. Suppose some other prime number $r | |F|$ and $r \neq p$.
  Then $F$ has some element of order $r$ under $+$ (by cauchy's theorem), say $b$. 
  Since $(r, p) = 1$, exists $s,t$ such that $sr + tp = 1$. So since $rb = pb = 0$, 
  $(sr + tp)b = b = 0$. Contradict that $b$ has order $r$. So factors of $|F|$ can only be 
  $p$, or $|F| = p^n$ for some $n$.
 \end{proof}

Now we look at some notable examples of rings. 
\begin{eg}[Matrix]
  Consider $2\times 2$ matrices over $\mathbb{Z}_p$. It is clear that it forms a 
  ring. A notable subring is the invertible matrices, 
  or say general linear group $GL_2(\mathbb{Z}_p)$. And we shall discuss the order of it. 

  It is equivalent to find the number of pairs such that $ad \equiv bc \mod p$ and minus it with $p^4$. 
  We consider the number of pairs $(a,d)$ s.t. $ad \equiv n$. We have $p-1$ pairs for $n \neq 0$, and 
  they are $(1, n), (2, 2^{-1} n), (3, 3^{-1} n), \ldots, (p-1, (p-1)^{-1} n)$. And it is 
  $(p-1) \cdot (p-1)^2$ in total for $n \neq 0$. 

  And for $n = 0$, we have $(k, 0), (0, k), (0, 0)$ for any $k \neq 0$, which is a total of $(2p-1)^2$.
  So it has an order of $p^4 - (p-1)^3 - (2p-1)^2 = p^4 - p^3 - p^2 + p$.
\end{eg}

\begin{eg}[Quaternions]
  Let $F$ be a field and consider the set of $a_1 + a_2 i + a_3 j + a_4 k$, say $Q$. And we 
  define $ij = k, jk = i, ki = j$ and $ji = -k, kj = -i, ik = -j$. We can show that 
  $Q$ is a non-commutative division ring. 
\end{eg}

\end{document}